% Document basé sur l'exemple generic-ornate-15min-45min.fr créé par Till Tantau <tantau@users.sourceforge.net> et traduit par Philippe De Sousa <philippejjg@free.fr>.
%https://github.com/josephwright/beamer/blob/main/doc/solutions/generic-talks/generic-ornate-15min-45min.fr.tex


\documentclass{beamer}

\mode<presentation> {
  \usetheme{Warsaw}
  %\setbeamercovered{transparent}	% donne un aperçu des élement cachés en transparence
}

\setbeameroption{show notes on second screen}

\usepackage[french]{babel}
\usepackage[utf8]{inputenc}
\usepackage[T1]{fontenc}

\usepackage{}

\title{Introduction à Git}

\author
{Olivier Morel \and S.~Lautre}
% - Composez les noms dans l'ordre dans lequel ils apparaîtrons dans l'article

\date[Version courte]
{Date / Cours du coaching}

\subject{Git}
% Inséré uniquement dans la page d'information du fichier PDF.


% TODO: Add logo AGE

% Si vous avez un fichier nommé "université-logo-nomfichier.xxx", où xxx
% est un format graphique accepté par latex ou pdflatex (comme par exemple .png),
% alors vous pouvez insérer votre logo ainsi :

% \pgfdeclareimage[height=0.5cm]{le-logo}{université-logo-nomfichier}
% \logo{\pgfuseimage{le-logo}}

% Si vous souhaitez recouvrir vos transparents un à un,
% utilisez la commande suivante (pour plus d'info, voir la page 74 du manuel
% d'utilisation de Beamer (version 3.06) par Till Tantau) :

%\beamerdefaultoverlayspecification{<+->}


\begin{document}
\begin{frame}
	\titlepage
\end{frame}

\begin{frame}{Sommaire}
	\tableofcontents
	% Vous pouvez, si vous le souhaiter ajouter l'option [pausesections]
\end{frame}

\section{Introduction}

\begin{frame}{Gestion de versions, qu'est ce que c'est?}

\end{frame}

\begin{frame}{Git}
	
\end{frame}

\begin{frame}{Git is not GitHub}

\end{frame}

\begin{frame}{Installation}
	
\end{frame}

\section{Création de repository}
\begin{frame}{Structure d'un repository}
	
\end{frame}

\begin{frame}{Créer un repository}
	
\end{frame}

\section{Travailler avec Git}
\begin{frame}{Les différents états de git}
	% TODO: add graph like this: https://ndpsoftware.com/git-cheatsheet.html#loc=workspace;
\end{frame}

\begin{frame}{Staging}
	
\end{frame}

\begin{frame}{Commiting}
	
\end{frame}

\begin{frame}{Branching}
	
\end{frame}

\begin{frame}{Merging}
	
\end{frame}

\section{Travailler avec GitHub}
\begin{frame}{GitHub}
	
\end{frame}

\begin{frame}{Ajouter un repository sur GitHub}

\end{frame}

\begin{frame}{Fetching}
	
\end{frame}

\begin{frame}{Pushing}
	
\end{frame}


\section{Résumé}
\begin{frame}{Résumé}

\end{frame}

\begin{frame}{Quelques liens utiles}
	
\end{frame}

\end{document}


% used as reference
\begin{frame}{Faîtes des titres qui informent}

	Vous pouvez créer des recouvrements\dots
	\begin{itemize}
	\item en utilisant la commande \texttt{pause} :
	  \begin{itemize}
	  \item
		Premier item.
		\pause
	  \item
		Second item.
	  \end{itemize}
	\item
	  en utilisant les spécifications des recouvrements
	  \begin{itemize}
	  \item<3->
		Premier item.
	  \item<4->
		Second item.
	  \end{itemize}
	\item
	  en utilisant en général la commande \texttt{uncover} :
	  \begin{itemize}
		\uncover<5->{\item
		  Premier item.}
		\uncover<6->{\item
		  Second item.}
	  \end{itemize}
	\end{itemize}
  \end{frame}